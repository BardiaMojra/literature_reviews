\documentclass[11pt]{article}
\usepackage{bookmark}
\usepackage{algorithm}
\usepackage{algpseudocode}
\usepackage{amsfonts}
\usepackage{amsmath}
\usepackage{amssymb}
\usepackage{amsthm}
\usepackage{bm}
\usepackage{color}
\usepackage{comment}
\usepackage{float}
\usepackage{graphicx}
%\usepackage[hidelinks]{hyperref}
\usepackage{makecell}
\usepackage[caption=false,font=footnotesize,subrefformat=parens,labelformat=parens]{subfig}
\usepackage{wrapfig}
\usepackage{url}
\usepackage[table]{xcolor}
%
\setlength{\parindent}{0.25in}
\setlength{\parskip}{.05in}
\pagestyle{plain}
%Title, date an author of the document
\title{Literature Review}
\author{Bardia Mojra}


\begin{document}
\maketitle
\thispagestyle{empty}

\bigskip
\bigskip
\begin{center}
      Robotic Vision Lab
\end{center}

\begin{center}
      The University of Texas at Arlington
\end{center}

\section{PVNet}

\begin{itemize}
      \item code: \url{}
      \item paper: \url{}
      \item citation:
\end{itemize}

      \subsection{Introduction}
      \par In this paper, the authors propose a novel two-stage pose estimation
      framework, Pixel-wise Voting Network or PVNet. First, they estimate 2D
      keypoints for each object in a RANSAC-like fashion which enables uncertainty
      measurement in the following stage. In the second stage, they use a
      modified EPnP algorithm \cite{lepetit2009epnp} that leverages feature
      uncertainty \cite{ferraz2014leveraging} to calculate estimated object 6D
      position.

      \subsection{Problem Statement}
      \par Although 6D pose estimation has been subject of research for many
      years and great accuracy has been achieved, many state of the art solutions
      do not take advantage of uncertainty among observed features.


      \subsection{Related Work}
      \subsubsection{Hollistic Methods}

      \subsubsection{Keypoint-Based Methods}

      \subsubsection{Dense Methods}

      \subsection{Method}
      \par This framework proposes a new method, PVNet, for generating
      keypoints uncertainty data. They

      mean and covariance information which they integrated into
      an the EPnP \cite{ferraz2014leveraging}





%Sets the bibliography style to UNSRT and import the
\newpage
\bibliography{references}
\bibliographystyle{ieeetr}

\end{document}
