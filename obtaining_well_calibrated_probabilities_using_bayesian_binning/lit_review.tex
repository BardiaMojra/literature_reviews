\documentclass[11pt]{article}
\usepackage{bookmark}
\usepackage{algorithm}
\usepackage{algpseudocode}
\usepackage{amsfonts}
\usepackage{amsmath}
\usepackage{amssymb}
\usepackage{amsthm}
\usepackage{bm}
\usepackage{color}
\usepackage{comment}
\usepackage{float}
\usepackage{graphicx}
%\usepackage[hidelinks]{hyperref}
\usepackage{makecell}
\usepackage[caption=false,font=footnotesize,subrefformat=parens,labelformat=parens]{subfig}
\usepackage{wrapfig}
\usepackage{url}
\usepackage[table]{xcolor}
%
\setlength{\parindent}{0.25in}
\setlength{\parskip}{.05in}
\pagestyle{plain}
%Title, date an author of the document
\title{Literature Review}
\author{Bardia Mojra}


\begin{document}
\maketitle
\thispagestyle{empty}

\bigskip
\bigskip
\begin{center}
      Robotic Vision Lab
\end{center}

\begin{center}
      The University of Texas at Arlington
\end{center}

\section{Obtaining Well Calibrated Probabilities Using Bayesian Binning}

\begin{itemize}
      \item code: \url{https://github.com/pakdaman/calibration/}
      \item paper: \url{https://www.ncbi.nlm.nih.gov/pmc/articles/PMC4410090/}
      \item citation: \cite{naeini2015obtaining}
\end{itemize}

      \subsection{Introduction}
      \par In this paper, the authors propose a novel calibration method for
      probabilistic predictive models. Bayesian Binning into Qualities (BBQ),
      is a non-parametric and post-processing calibration method. Their
      proposed method is a binary classifier calibration method is based on the
      histogram-binning calibration method \cite{zadrozny2001obtaining}. It is
      important to note this method could be extended to multi-class
      classification tasks \cite{zadrozny2002transforming}.

      \subsection{Problem Statement}
      \par In machine learning, classification problems are often solved by deploying a
      a predictive model trained on some given data set but often underperform and
      make miscalibrated predictions ("classifier score"). Statistically speaking, for
      a calibrated prediction of i.e. 40\%, there will be an occurrence of 40\% for a
      give test set that is 1) large enough with respect to solution space size, 2)
      test data set is randomly selected (for more insight read on
      \textit{Central Limit Theorem}.

      Figure 1 is a reliability curve \cite{degroot1983comparison}
      \cite{niculescu2005predicting} that is used as an example of a predictive
      model with poorly estimated probabilities.

      %\write18{wget https://www.ncbi.nlm.nih.gov/pmc/articles/PMC4410090/bin/nihms679964f1.jpg}
      %\includegraphics[with=\textwidth]{./nihms679964f1.jpg}

      \subsection{Related Work}
      \par Mainly, calibration is done two ways 1) \textit{ab initio}, by modifying
      objective function which increases computational cost. 2) It can be done as a
      post-processing procedure. Post processing can be categorized into parametric
      and non-parametric. Platt's method is an example of parametric calibration,
      \cite{platt1999probabilistic}. Non-parametric methods include histogram-
      binning \cite{zadrozny2001obtaining}, Platt scaling
      \cite{platt1999probabilistic}, and isotonic regression
      \cite{zadrozny2002transforming}.

      \subsection{Method}
      \par BBQ is an extension of simple histogram binning method
      \cite{zadrozny2001obtaining}, with added capability to consider different
      binning models (different number of bins) and their combinations under a
      Bayesian framework \cite{heckerman1995learning}. The generated Bayesian
      score provides further insight into network structure and it is also used
      when combining binning models.

      $$ \textit{Score}(M) ~=~ P(M)~.~P(D | M) $$

      The marginal likelihood, $P(D | M)$, has a closed form solution under the
      following 3 conditions, \cite{heckerman1995learning}:

      \begin{enumerate}
            \item All samples are under i.i.d. assumption and the class
            distribution $P(Z|B=b)$, which is class distribution for bin b,
            with a binomial distribution with parameter $\theta_b$.
            \item Bin distributions are independent.
            \item The prior distribution over binning model parameters $\theta$'s
            are modeled using a Beta distribution.
      \end{enumerate}

      Marginal likelihood in closed form, \cite{heckerman1995learning}:

      $$
      P(D|M) = \prod_{b=1}^{B} \frac{\Gamma(\frac{N'}{B})}{\Gamma (N_b + \frac{N'}{B})}
      \frac{\Gamma (m_b + \alpha_b)}{\Gamma (\alpha_b)}
      \frac{\Gamma (n_b + \beta_b)}{\Gamma (\beta_b)}
      $$

      Where:
      \begin{itemize}
      \item $\Gamma(n) = (n-1)!$
      \item $N_b$: The total number of training instances in the $b$'th bin.
      \item $n_b$: The total instances of class *zero* among all training
      instances $N_b$.
      \item $m_b$: The total instances of class *one* among all training
      instances $N_b$.
      \item $P(M)$: The prior distribution of binning model M, uniform
      distribution for initial condition.
      \end{itemize}

      \par The above equation is used for model averaging by BBQ, they point
      out that mentioned Bayesian scores could be used for model selection. Per
      \cite{hoeting1999bayesian}, model averaging is superior to model
      selection methods.

      \subsection{BBQ}
      BBQ framework defines calibrated prediction as:

      $$
      P(z=1|y) = \sum_{i=1}^{T} \frac{\textit{Score}(M_i)}{\sum_{j=1}^{T} \textit{Score}(M_j)}
      ~.~ P(z=1 | y, M_i)
      $$

      Where:
      \begin{itemize}
            \item $T$ : total number of binning models considered
            \item $P(z=1 | y, M_i)$ : probability estimate using model $M_i$ for uncalibrated
      classifier output y.
      \end{itemize}


      \subsection{Calibration Measures}
      \begin{itemize}
            \item ECE: Expected Calibration Error is calculated over the bins.
            \item MCE: Maximum Calibration Error is calculated among the bins.
      \end{itemize}

      $$
      ECE = \sum_{i=1}^{K}P(i).|o_i - e_i| ~~,~~ MCE= \max\limits_{i=1}^{K}(|o_i-e_i|),
      $$

      Where:
      \begin{itemize}
            \item $o_i$: true fraction of positive instances in the i$^{th}$ bin.
            \item $e_i$: mean of the post-calibrated probabilities in the i$^{th}$ bin.
            \item $P(i)$: empirical probability (fraction) of all instances in the i$^{th}$ bin.

      \end{itemize}
      \subsection{Empirical Results}
      \begin{itemize}
            \item Acc: accuracy
            \item AUC: area under the ROC curve (receiver operator characteristic curve).
            \item RMSE
            \item ECE
            \item MCE
      \end{itemize}


%Sets the bibliography style to UNSRT and import the
\newpage
\bibliography{references}
\bibliographystyle{ieeetr}

\end{document}
